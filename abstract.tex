% -*-mode: latex; coding: utf-8-unix -*-
\documentclass[11pt]{article}
\usepackage[utf8]{inputenc}
%
% user-modifiable options
%
\usepackage{leading} \leading{6mm} % extra space between lines

% \usepackage[francais,UKenglish]{babel}    % last one is default language
\usepackage[a4paper]{geometry}
% \usepackage{lmodern} %% Vector fonts
%
% Font and appearance: do not change
%
% \usepackage{aeguill} %% French guillemets with T1
\usepackage{amsfonts,amsmath,amssymb} %% Additional math chars
\usepackage{eurosym} %% \euro symbol
\usepackage[T1]{fontenc} %% Vector fonts
\usepackage[babel]{microtype} % load microtype after babel, inputenc, and font-changing packages
% \usepackage{flushend}	% balanced last page in 2-col formatting
%
% Useful packages.  Optional.
%
\usepackage{graphicx}
\usepackage[numbers,square,sort&compress]{natbib}
\usepackage[defblank]{paralist} % inline lists
\usepackage{parskip} \parskip=1ex plus 5pt minus 1pt \parindent=4ex %% indent paragraphs
\usepackage[obeyspaces]{url} \urlstyle{sf} %% URLs 
% \usepackage[verbatim]{svn-multi}
\usepackage{xcolor}
\usepackage{xspace}
\usepackage{hyperref}  %% hyperlinks in PDF (must be last package loaded)
    \newcommand{\hhref}[2]{\href{#2}{#1}} %{text}{URL}

% \documentclass[11pt]{article}
% \usepackage{inputenc}
% \usepackage{amssymb}
% \usepackage{graphicx}
% \usepackage{xcolor}

\newcommand\todo[1]{\textcolor{blue}{(TODO: #1)}}
\newcommand{\commentaire}[2][fromWhom?]{{
  \color{magenta}{\bfseries\sffamily\scriptsize$\triangleright$(#1:) #2$\triangleleft$}
}}


\begin{document}
\author{Saalik Hatia}
\title{Specification of persistent storage for a TCC database}
\date\today
\maketitle

\begin{abstract}

  % Large-scale application are typically built on top of geo-distributed
  % databases running on multiple datacenters (DCs) situated around the
  % globe.
  % Network failures are unavoidable, but in most internet services,
  % availability is required; however, the CAP theorem proves
  % that it is impossible to provide both availability and strong
  % consistency at the same time.
  % Sacrificing strong consistency, exposes developpers to complex anomalies
  % that are hard to build against.
  
  Checkpointing is a technique that is used in distributed 
  systems to help tolerate failures and speed up recovery for years.
  In AntidoteDB, a database that is geo-replicated and where data is 
  persisted in the form of a journal of operations, checkpointing is hard.

  The way AntidoteDB construct an object, is by reading all the updates 
  made on the object to create a version.
  This method allow the implementation of Multiversion concurrency
  control (MVCC) which enable multiple version of the same object to exist
  concurrently by reading the relevant operations.
  Making every update ever made an essential information to the database.

  In every datacenter (DC) data is sharded between several nodes called shards.
  Each shard handles a portion of the keyspace in the database thus having 
  each a journal of operations.
  All these updates at a DC are totally ordered whereas 
  updates at different locations are partially ordered. 
  Each DC is a full replica of the database and has an identical number of 
  shard and data can be updated at each location.

  The combination of multiple journals independently updated at different 
  locations, MVCC and partial order between DCs makes checkpointing and 
  truncation a challenge.

  The main objective of this work is to specify a mechanism for
  pruning the journal safely, by storing recent checkpoints.

  In each DC there is one journal per shard containing local updates, 
  each of these updates are sent to its counterpart in other DCs.
  Upon reception those updates remote DCs log them into a journal that 
  contains updates received from that location.
  Considering this at a single DC the number of journals amounts to the
  number of shards multiplied by the number of datacenters i.e. a replica.
  
  These journals grow without bound and in order to avoid storage issues 
  we need to truncate them.
  Before being able to truncate the journals safely, checkpoints need to be 
  created and stored durably.

  To properly choose a checkpointing cut we need to keep track
  of stateq throughout the multiple shards and further, multiple DCs.
  Updates can have dependencies in other shards, by being causally dependent or
  they are part of the same transaction, but can also come from remote DCs 
  making them present in replicated journals.

  First, we keep track of normal operations in the database.
  The persistent operations, the committed and in-flight transactions.
  By adding tracking of communication between shards in a datacenter, 
  and datacenters between them it allow us to create consistent cuts 
  that represent the state of the database through them.
  One of them represent the point of safety in a datacenter called 
  DC-wide causal safe point (DCSf) which ensures that all the updates 
  that are part of this cut are safe to read at a DC.
  
  Once we defined all these cuts our next focus is on interactions
  between all the modules that constitutes the system.
  To take the DCSf as an example it used to ensure that no client
  will read an transaction that is not safe.
  In this case for exemple the call fails and a signal is sent to the
  module responsable for flushing information to the disk.
  Using similar information we specify and write down the invariants 
  to ensures the system operates safely and is correct.
  To ensure correctness we will identify the invariants of each individual 
  component, the global nvariants and link them together. 
  Followed by writing the corresponding pseudo-code and formally verify it. 
  And finally implement and validate through experimentation.
  
  
  % AntidoteDB is a database designed for geo-replication.
  % It aims to provide high availability with the strongest possible
  % consistency model, it guarantees Transactional Causal Consistency (TCC)
  % and supports CRDTs.
  % TCC means that: (1) if one update happens before another, they will be
  % observed in the same order (causal consistency), and (2) updates in the
  % same transaction are observed in all-or-nothing fashion.
  
  % In AntidoteDB, the database is persisted as a journal of operations.
  % In the current implementation, the journal grows without bound.
  % The main objective of this work is to specify a mechanism for
  % pruning the journal safely, by storing recent checkpoints.
  
  % Data is sharded, each shard is responsible for a portion of the 
  % keyspace.
  % The way transactions are handled by the database is by contacting
  % each shard concerned by the operations in them.
  % Each shard in a DC is replicated to all other DCs. 
  % All updates that originate in some DC are sent asynchronously to the
  % corresponding shards in other DCs.
  % Making it a local update journal and replicated journals from every
  % miror shard in other DCs.

  % In order to properly choose a checkpointing cut we need to keep track
  % of the state throughout multiple shards and further, multiple DCs.
  % Because an update can have dependencies in other shards journal.

  % We use consistent cuts to represent states of durability and all the 
  % running and committed transactions. 
  % We build upon them to find a safe cut called DCSf that represent the 
  % safety point from which transaction can read from.  
  
  % A secondary objective is to be able to use multiple storage backends,
  % including legacy databases or file systems without change to their
  % native format.

  

  % To enable pruning we need to keep track of the overall state of the database.
  % With this in mind we study the design and properties of the current
  % architecture.
  % Using information at our disposal we maintain states to allow us
  % to keep track of the local state of the datacenter first.
  % Using communication between datacenters we are able to infer the 
  % global state of our system. 
  % Once we have these states we use them to characterise the operations and 
  % correctness of the system using invariants.

  

% \url{https://saalik.github.io/sharedfiles/specification.pdf}

\end{abstract}
\end{document}
