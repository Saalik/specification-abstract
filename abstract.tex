% -*-mode: latex; coding: utf-8-unix -*-
\documentclass[11pt]{article}
\usepackage[utf8]{inputenc}
%
% user-modifiable options
%
\usepackage{leading} \leading{6mm} % extra space between lines

% \usepackage[francais,UKenglish]{babel}    % last one is default language
\usepackage[a4paper]{geometry}
% \usepackage{lmodern} %% Vector fonts
%
% Font and appearance: do not change
%
% \usepackage{aeguill} %% French guillemets with T1
\usepackage{amsfonts,amsmath,amssymb} %% Additional math chars
\usepackage{eurosym} %% \euro symbol
\usepackage[T1]{fontenc} %% Vector fonts
\usepackage[babel]{microtype} % load microtype after babel, inputenc, and font-changing packages
% \usepackage{flushend}	% balanced last page in 2-col formatting
%
% Useful packages.  Optional.
%
\usepackage{graphicx}
\usepackage[numbers,square,sort&compress]{natbib}
\usepackage[defblank]{paralist} % inline lists
\usepackage{parskip} \parskip=1ex plus 5pt minus 1pt \parindent=4ex %% indent paragraphs
\usepackage[obeyspaces]{url} \urlstyle{sf} %% URLs 
% \usepackage[verbatim]{svn-multi}
\usepackage{xcolor}
\usepackage{xspace}
\usepackage{hyperref}  %% hyperlinks in PDF (must be last package loaded)
    \newcommand{\hhref}[2]{\href{#2}{#1}} %{text}{URL}

% \documentclass[11pt]{article}
% \usepackage{inputenc}
% \usepackage{amssymb}
% \usepackage{graphicx}
% \usepackage{xcolor}

\newcommand\todo[1]{\textcolor{blue}{(TODO: #1)}}
\newcommand{\commentaire}[2][fromWhom?]{{
  \color{magenta}{\bfseries\sffamily\scriptsize$\triangleright$(#1:) #2$\triangleleft$}
}}


\begin{document}
\author{Saalik Hatia}
\title{Specification of persistent storage for a TCC database}
\date\today
\maketitle

\begin{abstract}

  Large-scale application are typically built on top of geo-distributed
  databases running on multiple datacenters (DCs) situated around the
  globe.
  Network failures are unavoidable, but in most internet services,
  availability is required; however, the CAP theorem proves
  that it is impossible to provide both availability and strong
  consistency at the same time.
  Sacrificing strong consistency, exposes developpers to complex anomalies
  that are hard to build against.
  
  AntidoteDB is a database designed for geo-replication.
  It aims to provide high availability with the strongest possible
  consistency model, it guarantees Transactional Causal Consistency (TCC)
  and supports CRDTs.
  TCC means that: (1) if one update happens before another, they will be
  observed in the same order (causal consistency), and (2) updates in the
  same transaction are observed in all-or-nothing fashion.
  
  In AntidoteDB, the database is persisted as a journal of operations.
  In the current implementation, the journal grows without bound.
  The main objective of this work is to specify a mechanism for
  pruning the journal safely, by storing recent checkpoints.
  This will enable faster reads and crash recovery.
  A secondary objective is to be able to use multiple storage backends,
  including legacy databases or file systems without change to their
  native format.

  To enable pruning we need to keep track of the overall state of the database.
  With this in mind we study the design and properties of the current
  architecture.
  Using information at our disposal we maintain states to allow us
  to keep track of the local state of the datacenter first.
  Using communication between datacenters we are able to infer the 
  global state of our system. 
  Once we have these states we use them to characterise the operations and 
  correctness of the system using invariants.

  

% \url{https://saalik.github.io/sharedfiles/specification.pdf}

\end{abstract}
\end{document}
