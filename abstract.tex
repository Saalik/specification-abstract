% -*-mode: latex; coding: utf-8-unix -*-
\documentclass[11pt]{article}
\usepackage[utf8]{inputenc}
%
% user-modifiable options
%
\usepackage{leading} \leading{6mm} % extra space between lines

% \usepackage[francais,UKenglish]{babel}    % last one is default language
\usepackage[a4paper]{geometry}
% \usepackage{lmodern} %% Vector fonts
%
% Font and appearance: do not change
%
% \usepackage{aeguill} %% French guillemets with T1
\usepackage{amsfonts,amsmath,amssymb} %% Additional math chars
\usepackage{eurosym} %% \euro symbol
\usepackage[T1]{fontenc} %% Vector fonts
\usepackage[babel]{microtype} % load microtype after babel, inputenc, and font-changing packages
% \usepackage{flushend}	% balanced last page in 2-col formatting
%
% Useful packages.  Optional.
%
\usepackage{graphicx}
\usepackage[numbers,square,sort&compress]{natbib}
\usepackage[defblank]{paralist} % inline lists
\usepackage{parskip} \parskip=1ex plus 5pt minus 1pt \parindent=4ex %% indent paragraphs
\usepackage[obeyspaces]{url} \urlstyle{sf} %% URLs 
% \usepackage[verbatim]{svn-multi}
\usepackage{xcolor}
\usepackage{xspace}
\usepackage{hyperref}  %% hyperlinks in PDF (must be last package loaded)
    \newcommand{\hhref}[2]{\href{#2}{#1}} %{text}{URL}

% \documentclass[11pt]{article}
% \usepackage{inputenc}
% \usepackage{amssymb}
% \usepackage{graphicx}
% \usepackage{xcolor}

\newcommand\todo[1]{\textcolor{blue}{(TODO: #1)}}
\newcommand{\commentaire}[2][fromWhom?]{{
  \color{magenta}{\bfseries\sffamily\scriptsize$\triangleright$(#1:) #2$\triangleleft$}
}}


\begin{document}
\author{Saalik Hatia}
\title{Specification of persistent storage for a TCC database}
\date\today
\maketitle

\begin{abstract}

In the context of the CAP theorem Transactional Causal Consistency (TCC) 
is a transactional model that provides availability with the strongest 
possible consistency model. TCC means that: 
(1) if one update happens before another, they will be
observed in the same order (causal consistency), and (2) updates in the
same transaction are observed all-or-nothing.

In a database implementing TCC, data is persisted as a journal of operations.
This journal must be pruned or it will grow without bound.
In order to do so safely, the database needs to safely store 
checkpoints.
The specification of this mechanism must ensure that checkpoints are a TCC
cut of the data store, and that all updates that are truncated are available
through a checkpoint store.

The objective of this work is to present this specification.

\url{https://saalik.github.io/sharedfiles/specification.pdf}

\end{abstractt}
\end{document}
